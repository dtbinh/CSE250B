\documentclass[a4paper,11pt]{article}

\usepackage[T1]{fontenc}
\usepackage[utf8]{inputenc}
\usepackage{graphicx}
\usepackage{xcolor}
 \usepackage{tgtermes}
\usepackage{listings}
\usepackage{minted}
 \usepackage[
 pdftitle={Math Assignment},
 pdfauthor={Joe Doe, Some University},
 colorlinks=true,linkcolor=blue,urlcolor=blue,citecolor=blue,bookmarks=true,
 bookmarksopenlevel=2]{hyperref}
\usepackage{amsmath,amssymb,amsthm,textcomp}
\usepackage{enumerate}
\usepackage{multicol}
\usepackage{tikz}

\usepackage{geometry}
\geometry{total={210mm,297mm},
left=25mm,right=25mm,%
bindingoffset=0mm, top=20mm,bottom=20mm}


\linespread{1.3}

\newcommand{\linia}{\rule{\linewidth}{0.5pt}}

% custom theorems if needed
\newtheoremstyle{mytheor}
    {1ex}{1ex}{\normalfont}{0pt}{\scshape}{.}{1ex}
    {{\thmname{#1 }}{\thmnumber{#2}}{\thmnote{ (#3)}}}

\theoremstyle{mytheor}
\newtheorem{defi}{Definition}
\usepackage[ruled, vlined, linesnumbered,lined,boxed,commentsnumbered]{algorithm2e}
\usepackage[parfill]{parskip}
\makeatletter

\setlength\parindent{0pt}
% custom footers and headers
\usepackage{fancyhdr,lastpage}


\newcommand{\myequ}[1]{\begin{align}\begin{split} #1 \end{split}\end{align}}


\begin{document}

\title{CSE 250B: Machine Learning}

\author{Sai Bi}

\date{\today}

\maketitle

\section*{Problem 1}
\subsection*{a}
$Mv_i = \sigma_i u_i$

\subsection*{b}
$M^Tu_i = \sigma_i v_i$

\subsection*{c}
$M^T Mv_i = \sigma_i M^T u_i = \sigma_i^2 v_i$

$MM^Tu_i = \sigma_i M v_i = \sigma_i^2 u_i$ 

\subsection*{d}
The eigenvalues of $MM^T$ is $\sigma_1^2, \sigma_2^2, ..., \sigma_p^2$, and the corresponding 
eigenvectors are $u_1, u_2, ..., u_p$.

\subsection*{e}
$MM^T$ and $M^TM$ has the same set of eigenvalues, but different eigenvectors.

\subsection*{f}
There are exactly $k$ non-zero $\sigma_i$.

\section*{Problem 2}
\subsection*{a}
The best rank-$1$ approximation is:
\begin{align}
\begin{split}
\begin{bmatrix}
    1.5745   & 2.0801  &  2.5857 \\
    3.7594   & 4.9664  &  6.1735 \\
\end{bmatrix}
\end{split}
\end{align}

\subsection*{b}
The decomposition is not unique.
Let $A$ be the rank-$1$ matrix as following:
\begin{align}
\begin{split}
A = \begin{bmatrix}
1  &0  &0 \\
2  &0  &0 \\
3  &0  &0 \\
\end{bmatrix}
\end{split}
\end{align}
then we have 
\begin{align}
\begin{split}
a = \begin{bmatrix}
1  \\
2  \\
3  \\
\end{bmatrix}
\hspace{0.5cm}
b = \begin{bmatrix}
1  \\
0  \\
0  \\
\end{bmatrix}
\hspace{0.5cm}
u = \begin{bmatrix}
2  \\
4  \\
6  \\
\end{bmatrix}
\hspace{0.5cm}
v = \begin{bmatrix}
0.5  \\
0  \\
0  \\
\end{bmatrix}
\end{split}
\end{align}

satisfying $uv^T = ab^T = A$. Therefore, the decomposition is not unique.

\subsection*{c}
$\widehat{M} = \sum\limits_{i=1}^{k}\sigma_i u_i v_i^T$




\section*{Problem 3}
\subsection*{a}
The Gram matrix of the dataset is:
\begin{align}
    \begin{split}
        \begin{bmatrix}
        1   &  1 &    1   &  1 \\
        1   &  2   &  1    & 2 \\
        1    & 1    & 2    & 2 \\
        1   &  2   &  2   &  3 \\
        \end{bmatrix}
    \end{split}
\end{align}

There are other sets of four points with exactly the same Gram matrix, for example:
\begin{align}
\begin{split}
(-1, 0,0), (-1, 0, -1), (-1, -1, 0), (-1, -1, -1)
\end{split}
\end{align}

\section*{Problem 4}
\subsection*{b}
See Figure~\ref{fig:4b0}. The plot doesn't look right since right and left, top and bottom are reversed.

To fix the problem, just multiply the $x-y$ coordinates of each point by $-1$. After the problem is fixed,
the results is Figure~\ref{fig:4b1}.

\subsection*{c}


\begin{figure}[h]
    \centering{
        \includegraphics[width=0.9\textwidth]{./code/4b-0.png}  
    }
    \caption{Classical multidimensional scaling results.}
    \label{fig:4b0}
\end{figure}


\begin{figure}[h]
    \centering{
        \includegraphics[width=0.9\textwidth]{./code/4b-1.png}  
    }
    \caption{Classical multidimensional scaling results after fixing the problem.}
    \label{fig:4b1}
\end{figure}




















\end{document}
